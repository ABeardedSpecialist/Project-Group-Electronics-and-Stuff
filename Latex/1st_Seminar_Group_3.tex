%----------------------------------------------------------------------------------------
%
% LaTeX-template for degree projects at LNU, Department of Computer Science
% Last updated by Johan Hagelbäck, Oct 2015
% Linnaeus University
%
% License: Creative Commons BY
%
%----------------------------------------------------------------------------------------

%----------------------------------------------------------------------------------------
%	Settings and configuration
%----------------------------------------------------------------------------------------

\documentclass[a4paper,12pt]{article}

\usepackage[T1]{fontenc}
\usepackage{times}
\usepackage[english]{babel}
\usepackage[utf8]{inputenc}
\usepackage{wallpaper}
\usepackage[absolute]{textpos}
\usepackage[top=2cm, bottom=2.5cm, left=3cm, right=3cm]{geometry}
\usepackage{appendix}
\usepackage[nottoc]{tocbibind}
\usepackage{enumerate}


\setcounter{secnumdepth}{3}
\setcounter{tocdepth}{3}

\usepackage{sectsty}
\sectionfont{\fontsize{14}{15}\selectfont}
\subsectionfont{\fontsize{12}{15}\selectfont}
\subsubsectionfont{\fontsize{12}{15}\selectfont}
\usepackage[font=large, labelfont=bf]{caption}

\usepackage{csquotes} % Used to handle citations

\renewcommand{\thetable}{\arabic{section}.\arabic{table}}  
\renewcommand{\thefigure}{\arabic{section}.\arabic{figure}} 

%----------------------------------------------------------------------------------------
%	
%----------------------------------------------------------------------------------------
\newsavebox{\mybox}
\newlength{\mydepth}
\newlength{\myheight}

\newenvironment{sidebar}%
{\begin{lrbox}{\mybox}\begin{minipage}{\textwidth}}%
{\end{minipage}\end{lrbox}%
 \settodepth{\mydepth}{\usebox{\mybox}}%
 \settoheight{\myheight}{\usebox{\mybox}}%
 \addtolength{\myheight}{\mydepth}%
 \noindent\makebox[0pt]{\hspace{-20pt}\rule[-\mydepth]{1pt}{\myheight}}%
 \usebox{\mybox}}

%----------------------------------------------------------------------------------------
%	Title section
%----------------------------------------------------------------------------------------
\newcommand\BackgroundPic{
    \put(-2,-3){
    \includegraphics[keepaspectratio,scale=0.3]{img/lnu_etch.png} % Background picture
    }
}
\newcommand\BackgroundPicLogo{
    \put(30,740){
    \includegraphics[keepaspectratio,scale=0.10]{img/logo.png} % Logo in upper left corner
    }
}

\title{	
\vspace{-8cm}
\begin{sidebar}
    \vspace{10cm}
    \normalfont \normalsize
    \Huge Report \\
    \vspace{-1.3cm}
\end{sidebar}
\vspace{3cm}
\begin{flushleft}
    \huge Project Course In Computer Science\\ 
    \it \LARGE - Team 3 first Seminar
\end{flushleft}
\null
\vfill
\begin{textblock}{6}(10,13)
\begin{flushright}
\begin{minipage}{\textwidth}
\begin{flushleft} \large
\emph{Author:} Quasim Aljubarah, Michael Johansson, Tadas Lisauskas, Zeyuan Li, Robin Reijo and Robin Stempa\\ % Author
\emph{Supervisor:} Ola Petersson\\ % Supervisor
%\emph{Examiner:} Dr.~Mark \textsc{Brown}\\ % Examiner (course manager)
\emph{Semester:} VT 2016\\ % 
\emph{Subject:} 1DV508\\ % Subject area
\end{flushleft}
\end{minipage}
\end{flushright}
\end{textblock}
}

\date{} 

\begin{document}
\pagenumbering{gobble}
\newgeometry{left=5cm}
\AddToShipoutPicture*{\BackgroundPic}
\AddToShipoutPicture*{\BackgroundPicLogo}
\maketitle
\restoregeometry
\clearpage

\selectlanguage{english}

\tableofcontents

\newpage
\pagenumbering{gobble}
\pagenumbering{arabic}

%----------------------------------------------------------------------------------------
%
%	Here follows the actual text contents of the report.
%
%----------------------------------------------------------------------------------------
\section{Introduction}
Second week's to-do list was to update the first weeks  website design and to update the document with functionality/analysis for the webshop. 

\section{Products And Categories}
We have to main category that is "Computers and misc" where we are going to sell some pre built computers and laptops, some monitors and also hardware/software so our categories on the page will be the following: 
\begin{itemize}
	\item{Monitors}
	\subitem{less than 24"}
	\subitem{24-27"}
	\subitem{larger than 27"}
	
	\item{Computers}
	\subitem{Asus}
	\subitem{MSI}
	\subitem{etc.}
	
	\item{Laptops}
	\subitem{Asus}
	\subitem{MSI}
	\subitem{etc.}
	
	\item{Computer Components}
	\subitem{Graphics cards}
		\subitem{Hard drives}
		\subsubitem{SSD}
		\subsubitem{Mechanical}
	\subitem{Processors}
	\subitem{etc.}
	
	\item{Software}
	\subitem{Windows}
	\subitem{Anti-virus}
	\subitem{Adobe}
	
	\item{Accessories}
	\subitem{Keyboards}
	\subitem{Mice}
	\subitem{Headphones}
	\subitem{etc.}
	
\end{itemize}

\section{Website Design Sketch}
The website will be similar to most of the web shops. Having a search bar, products/categories list, add to cart button, checking every product, adding to cart and administrator login button.

Administrator view will have extra buttons on all of the same pages and some of the bars changed to edit (instead of add to cart).

\subsection{Regular View}
\begin{figure}[htbp]
	\caption{Homepage}
	\includegraphics[width=\textwidth,height=\textheight,keepaspectratio]{img/Homepage.png}
\end{figure}

\begin{figure}[htbp]
	\caption{Product view}
	\includegraphics[width=\textwidth,height=\textheight,keepaspectratio]{img/Product_view.png}
\end{figure}

\begin{figure}[htbp]
	\caption{Checkout}
	\includegraphics[width=\textwidth,height=\textheight,keepaspectratio]{img/Checkout.png}
\end{figure}

\begin{figure}[htbp]
	\caption{Search}
	\includegraphics[width=\textwidth,height=\textheight,keepaspectratio]{img/Search.png}
\end{figure}

\newpage
\subsection{Administrator view}

\begin{figure}[htbp]
	\caption{Admin homepage}
	\includegraphics[width=\textwidth,height=\textheight,keepaspectratio]{img/Admin_Homepage.png}
\end{figure}

\begin{figure}[htbp]
	\caption{Admin product view}
	\includegraphics[width=\textwidth,height=\textheight,keepaspectratio]{img/Admin_product.png}
\end{figure}

\begin{figure}[htbp]
	\caption{Admin order management}
	\includegraphics[width=\textwidth,height=\textheight,keepaspectratio]{img/Admin_order.png}
\end{figure}

\begin{figure}[htbp]
	\caption{Admin products view}
	\includegraphics[width=\textwidth,height=\textheight,keepaspectratio]{img/Products_admin.png}
\end{figure}
\newpage
\section{Analysis}
This section will cover all the requirements for the webshop and define them.
\subsection{General requirements }
This table analyses the general requirements for the webshop.
\newline

\begin{table}[htbp]
	\centering
	\caption{My caption}
	\label{my-label}
	\begin{tabular}{|l|l|}
		\hline
		\textbf{ID}    & \textbf{Requirements}                                                                                                               \\ \hline
		\textbf{1}     & The webshop must contain any number of products.                                                                                    \\ \hline
		\textbf{1.1}   & All information about the products shall be stored in the database.                                                                 \\ \hline
		\textbf{1.1.1} & \begin{tabular}[c]{@{}l@{}}product must have:\\ -Product name\\ -Category\\ -Quantity\\ -Price\\ -Description\\ -Image\end{tabular} \\ \hline
		\textbf{1.1.2} & All products will be pre-defined to categories (Monitors,Computers, Etc.)                                                           \\ \hline
		\textbf{1.2}   & When an order is placed it must be added to the database                                                                            \\ \hline
		\textbf{1.2.1} & When an order is placed the quantity of the products must change                                                                    \\ \hline
	\end{tabular}
\end{table}
\subsection{Customer requirements}
This table will look into the customer requirements
\newline
\begin{table}[htbp]
	\centering
	\caption{My caption}
	\label{my-label}
	\begin{tabular}{|l|l|}
		\hline
		\textbf{ID}    & \textbf{Requirement}                                                                                                                                                                          \\ \hline
		\textbf{2}     & \begin{tabular}[c]{@{}l@{}}The customer must be able to browse the products in the following ways:\\ -Browse all products\\ -Search for a product\\ -Show products in a category\end{tabular} \\ \hline
		\textbf{2.1}   & It must be possible to click on a product to see more details about it.                                                                                                                       \\ \hline
		\textbf{2.2}   & The customer must be able to put a product in a cart.                                                                                                                                         \\ \hline
		\textbf{2.2.1} & The customer must be able to view the products in the cart.                                                                                                                                   \\ \hline
		\textbf{2.2.2} & The customer must be able to change the quantity of a product in the cart.                                                                                                                    \\ \hline
		\textbf{2.2.3} & \begin{tabular}[c]{@{}l@{}}The customer must be able to remove a product from the cart with one click with any \\ quantity.\end{tabular}                                                      \\ \hline
		\textbf{2.3}   & The customer must be able to place an order   .                                                                                                                                               \\ \hline
		\textbf{2.3.1} & \begin{tabular}[c]{@{}l@{}}When placing an order, a form must be shown where the customer enters his/her \\ contact details (email, address, phone number).\end{tabular}                      \\ \hline
		\textbf{2.3.2} & When an order is finalized the customer must receive an auto-generated order number.                                                                                                          \\ \hline
		\textbf{2.3.3} & The customer must be able to check the status of his/her order from the order number.                                                                                                         \\ \hline
	\end{tabular}
\end{table}
\subsection{Administrator requirements}
Last section will look into the requirements for the administrator view mode and the functionalities it must have.
\newline


\begin{table}[htpb]
	\centering
	\caption{Administrator requirements table}
	\label{my-label}
	\begin{tabular}{|l|l|}
		\hline
		\textbf{ID}    & \textbf{Requirement}                                                       \\ \hline
		\textbf{3}     & The webshop shall have an Admin system where an administrator can log on   \\ \hline
		\textbf{3.1}   & Administrator accounts and passwords shall be stored in the database.      \\ \hline
		\textbf{3.2}   & An administrator must be able to add a new product                         \\ \hline
		\textbf{3.2.1} & An administrator must be able to remove a product                          \\ \hline
		\textbf{3.2.2} & An administrator must be able to change information about a product        \\ \hline
		\textbf{3.2.3} & An administrator must be able to add a category                            \\ \hline
		\textbf{3.2.4} & An administrator must be able to remove a category                         \\ \hline
		\textbf{3.2.5} & An administrator must be able to change information about a category       \\ \hline
		\textbf{3.3}   & An administrator must be able to add a new administrator account           \\ \hline
		\textbf{3.3.1} & An administrator must be able to remove a administrator account            \\ \hline
		\textbf{3.4}   & An administrator must be able to view all orders                           \\ \hline
		\textbf{3.4.1} & An administrator must be able update the order status on every order       \\ \hline
		\textbf{3.4.2} & The order status can only be New, Shipped, Delayed, Delivered and Returned \\ \hline
	\end{tabular}
\end{table}

\subsection{UML Design}
The preliminary UML design is in the picture below.
\begin{figure}[htbp]
	\caption{UML Design}
	\includegraphics[width=\textwidth,height=\textheight,keepaspectratio]{img/UML.png}
\end{figure}
\end{document}
